Tendo em consideração o objetivo para a implementação
desta fase, foi definido um conjunto de requisitos que
melhor representam o que se pretende.
Ambos o motor gráfico e o gerador de primitivas terão
de ser atualizados.

\subsection{Patches de Bezier}

Será necessário, agora, ao gerador de primitivas
conseguir gerar superfícies curvas através de
\textbf{patches de Bezier}.
\newline
\break
\noindent
Para tal, o gerador de primitivas deverá receber
como argumento um conjunto de \textbf{patches} de
pontos que consiguam definir uma primitiva, assim como,
um nível de \textbf{Tesselation} que defina a curvatura
de cada superfície.
\newline
\break
\noindent
Este conjunto de \textbf{patches} deverá, por simplicidade,
ser dado como forma de um único ficheiro onde serão descritos
os pontos que os consituem, assim como, os próprios
\textbf{patches}.
\newline
\break
\noindent
Estes ficheiros deverão, então, seguir o seguinte formato:
\newline

\begin{tcolorbox}[
    colback=blue!10!white,
    colframe=black!50!black,
]
\begin{verbatim}
2 <- número de patches
0, 1, 2, 3, 4, 5, 6, 7, 8, 9, 10, 11, 12, 13, 14, 15 <- ind.
3, 16, 17, 18, 7, 19, 20, 21, 11, 22, 23, 24, 15, 25, 26, 27
28 <- número de pontos de controlo
1.4, 0, 2.4 <- ponto de controlo 0
1.4, -0.784, 2.4 <- ponto de controlo 1
0.784, -1.4, 2.4 <- ponto de controlo 2
0, -1.4, 2.4
1.3375, 0, 2.53125
1.3375, -0.749, 2.53125
0.749, -1.3375, 2.53125
0, -1.3375, 2.53125
1.4375, 0, 2.53125
1.4375, -0.805, 2.53125
0.805, -1.4375, 2.53125
0, -1.4375, 2.53125
1.5, 0, 2.4
1.5, -0.84, 2.4
0.84, -1.5, 2.4
0, -1.5, 2.4
-0.784, -1.4, 2.4
-1.4, -0.784, 2.4
-1.4, 0, 2.4
-0.749, -1.3375, 2.53125
-1.3375, -0.749, 2.53125
-1.3375, 0, 2.53125
-0.805, -1.4375, 2.53125
-1.4375, -0.805, 2.53125
-1.4375, 0, 2.53125
-0.84, -1.5, 2.4 <- ponto de controlo 26
-1.5, -0.84, 2.4 <- ponto de controlo 27
\end{verbatim}
\end{tcolorbox}

\vspace{12pt}
\noindent
O argumento deste patch deverá, portanto, ser dado como:

\begin{lstlisting}[style=BASH]
    $ generator patch path/to/file.patch tesselation path/to/file.3d
\end{lstlisting}

\subsection{Transformações Animadas}
Para além destas novas primitivas, deverão, ainda,
ser implementadas animações sobre a forma de transformações.
\newline
\break
\noindent
Estas transformações, em constrate com as definidas na fase
anterior, deverão variar consoante o tempo decorrido, recebendo,
portanto, como parâmetro, um tempo para a sua execução.
Apenas as rotações e as translações irão ser capazes de
animação.
\newline
\break
\noindent
As translações deverão ser, então, animadas através de uma
curva de \textbf{Catmull-Rom}, seguindo, em ciclo, durante
o intervalo de tempo definido, um conjunto de \textbf{quatro
ou mais} pontos dado como argumento.
\newline
\break
\noindent
Já as rotações serão animadas por uma rotação constante de
360º que terá duração do intervalo de tempo definido.
\newline
\break
\noindent
Estas novas transformações deverão, como presumível, ser
definidas através de uma configuração \textbf{XML} do seguinte
formato:
\newline

\begin{tcolorbox}[
    colback=blue!10!white,
    colframe=black!50!black,
    after upper={\hfill\textbf{xml}}
]
\begin{verbatim}
    ...
    <translate time="10" align="True" >
        <point x="1" y="0" z="1" />
        <point x="0.707" y="0.707" <="1”"/>
        <point x="0" y="1" z="1" />
        ...
        <point x="-1" y="0" z="1" />
    </translate>
    ...

    ...
    <rotate time="10" x="0" y="1" z="0" />
    ...
\end{verbatim}
\end{tcolorbox}

\subsection{Representação por VBO}
Por último, deverá ser tido, ainda, em atenção
da necessidade de trocar o modo imediato de representação
para a representação por \textbf{VBO}, o que trará
melhorias de desempenho à solução.