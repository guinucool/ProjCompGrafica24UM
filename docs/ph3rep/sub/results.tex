Concluída a solução, foram construídas algumas cenas
tridimensionais, como de costume, que pretendem testar e exemplificar as funcionalidades
implementadas na solução obtida.
\newline
\break
\noindent
Estas cenas pretendem demonstrar o uso de \textbf{patches}
e transformações animadas de forma a melhorar modelos já construídos anteriormente.
\newline
\break
\noindent
Estas cenas podem, então, ser observados nas figuras
\ref{fig:snowman}, \ref{fig:solarlign} e \ref{fig:solar}.

\subsection{Boneco de Neve}

\begin{center}
    \includegraphics[width=0.8\textwidth]{imgs/boneconeve.png}
    \captionof{figure}{Boneco de Neve construído com transformações}
    \label{fig:snowman}
\end{center}

\subsection{Sistema Solar Alinhado}

\begin{center}
    \includegraphics[width=0.8\textwidth]{imgs/sissolarhor.png}
    \captionof{figure}{Sistema Solar construído com transformações}
    \label{fig:solarlign}
\end{center}

\subsection{Sistema Solar}

\begin{center}
    \includegraphics[width=0.8\textwidth]{imgs/sissolar.png}
    \captionof{figure}{Sistema Solar disperso construído com transformações}
    \label{fig:solar}
\end{center}

