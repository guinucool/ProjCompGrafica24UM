Definida a nova arquitetura e tendo as necessidades de
implementação em mente, foram, agora, tomadas algumas
decisões da implementação da arquitetura modificada.
\newline
\break
\noindent
Como especificado anteriormente, a arquitetura da 
transformação irá fazer uso de uma super-classe, que
deverá especificar que todas as suas sub-classes possuem
um método que permite à transformação definida ser
aplicada.
\newline
\break
\noindent
Estas transformações deverão, então, ser guardadas numa
lista de transformações do grupo de forma a manter a sua
ordem de aplicação (uma vez que o resultado é diferente
dependendo da ordem das transformações), onde não deverá
haver duas transformações do mesmo tipo.
\newline
\break
\noindent
Já a definição dos sub-grupos seguirá uma estratégia
semelhante, guardando os vários sub-grupos numa lista
de grupos. Cada grupo será, então, desenhado pela ordem
de aparição na configuração, fazendo uso da definição
recursiva para a função de desenho.
\newline
\break
\noindent
Por último, de forma a manter a herança de
transformações e evitar a sua aplicação indesejada,
o início do desenho de cada grupo será
o armazenamento da matriz atual, seguido da aplicação
das transformações, desenho dos modelos e dos grupos e,
posteriormente, de um restauro da matriz armazenada
inicialmente.