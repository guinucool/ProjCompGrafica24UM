\section{Conceptualização}

De forma a atingir os objetivos propostos, foi necessário desenvolver um sistema
dividido em Motor Grafico e Gerador de Primitivas. O Motor Gráfico é responsável
por gerir e renderizar a cena e os modelos geométricos, enquanto que o Gerador de Primitivas
é responsável por criar as primitivas geométricas que compõem os modelos. 


\subsection{Modelo de Domínio}

\includegraphics[width=15cm]{ModeloDominio.png}\\[\bigskipamount]

\subsection{Motor Gráfico}

Para atribuir uma organização ao sistema, foi decidido atribuir ao Motor Gráfico a
responsabilidade de ler ficheiros XML que descrevem a cena e os modelos, ao que chamamos 
World. O World é composto por um grupo de primitivas, que por sua vez são
desenhados na prespetiva de uma câmara e uma janela. Todos os objetos contidos no World
estão no ficheiro XML assim como as propriedades da câmara e da janela.
Quando o Motor Grafico renderiza o World, este necessita de ler os ficheiros .3d que
contêm as primitivas geométricas. Estes ficheiros são criados pelo Gerador de Primitivas

\subsection{Gerador de Primitivas}

O Gerador de Primitivas é responsável por criar as primitivas geométricas que serão
transformadas em Ficheiros .3d e posteriormente lidos pelo Motor Gráfico
quando este renderiza a cena. Quando é invocado, este gera um ficheiro .3d para a
primitiva geométrica que é passada como argumento. Este ficheiro contém a descrição
da primitiva geométrica em formato binário, que é lido pelo Motor Gráfico durante a
renderização da cena.

