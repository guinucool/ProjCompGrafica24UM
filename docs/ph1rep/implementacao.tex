\section{Implementação}

Definida uma estrutura base para a arquitetura do projeto,
determinando-se como cada parte da estrutura construída ia ser
implementada na solução final.

\subsection{Ficheiros .3d}

Tomou-se como ponto de partida a definição da estrutura dos ficheiros que
vão ser usados para comunicação das primitivas entre o gerador e o motor
gráfico.\newline
\break
\noindent
Decidiu-se, portanto, por questões de facilidade e eficiência de leitura
e escrita, usar uma solução binária para a composição destes
ficheiros.\newline
\break
\noindent
Cada ponto deverá escrever as suas coordenadas em formato de
\textit{float} binário, ocupando no ficheiro 4 bytes por
valor de coordenada, ou, na totalidade, 12 bytes pelas três
coordenadas x, y, z de cada ponto.\newline
\break
\noindent
Cada três pontos definirão uma face, originando um custo de 36 bytes
por face, e um conjunto de faces representarão uma primitiva,
produzindo um custo de NrFaces * 36 bytes por ficheiro.\newline
\break
\noindent
Em ficheiro, este formato pode, portanto, ser visto, considerando
os espaços e os parágrados como inexistentes, da seguinte forma:\newline

\begin{tcolorbox}[
    colback=gray!10!white,
    colframe=black!50!black,
    after upper={\hfill\textbf{.3d simplificado}}
]
\begin{verbatim}
0.0000 0.0000 0.0000 (em binário) | Face1
...Ponto2...                      | Face1
...Ponto3...                      | Face1
...Ponto4...                      | Face2
...Ponto5...                      | Face2
...Ponto6...                      | Face2
...Ponto7...                      | Face3
...                               ...
\end{verbatim}
\end{tcolorbox}

\break
\noindent
Embora não seja o principal motivo, esta estrutura binária também irá
permitir uma redução do espaço que estes ficheiros ocuparão no disco em
comparação a uma solução baseada num formato textual.\newline

\subsection{Gerador de primitivas}

Estipulado o formato dos ficheiros que deverão ser usados para armazenar
e representar primitivas, é necessário definir como o gerador irá criar e
guardar estes modelos de acordo com a estrutura decidida.\newline
\break
\noindent
Para a sua implementação, foi decidido usar uma abordagem por matrizes
para a representação dos pontos, aproveitando a eficiência de cálculo
que estas trazem para a aplicação de transformações aos pontos que irão
constituir as primitivas.\newline
\break
\noindent
Como deliberado na conceptualização, o gerador será composto por uma
hierarquia de primitivas que contêm faces, que, por sua vez, contêm
pontos. Esta arquitetura trará bastantes vantagens para a construção
de modelos, permitindo aplicar transformações a vários pontos que estejam
contidos numa face (ou numa primitiva) de uma só vez.\newline
\break
\noindent
Este conceito permite a possibilidade de ver as primitivas como uma
espécie de conjunto de sub-primitivas que se repetem ao longo do modelo
com diferentes posições e rotações.\newline
\break
\noindent
Assim sendo, é possível criar uma primitiva através de várias cópias
de uma ou mais sub-primitivas aplicando-lhes diferentes translações
e rotações de forma a obter um modelo definitivo final.\newline
\break
\noindent
Esta estratégia permitirá, então, a criação de primitivas através de
uma simples inicialização de elementos comuns na estrutura e posterior
aplicação de diferentes transformações a várias cópias destes, evitando
a criação singular de cada ponto necessária à representação final
da figura tridimensional.\newline
\break
\noindent
Ainda se decidiu implementar métodos de auxilio à criação destas
matrizes de translação, rotação e pontos, existindo a hipótese de
gerar pontos e translações através de coordenadas polares, que serão
bastante úteis para a representação do cone, por exemplo.

\subsection{Primitivas}

Aproveitando o conceito definido anteriormente para a definição de
primitivas, definiu-se, de forma simples, como poderiam os modelos
pretendidos ser, então, gerados.

\subsubsection{Plano}

Usando como ponto de partido o modelo mais simples (e que virá auxiliar
na criação de um outro modelo), o processo de definição deste começa pela
perceção do plano como um conjunto de planos mais pequenos, no caso de
as divisões pretendidas serem superior a uma.\newline
\break
\noindent
Esta visão permite, portanto, definir o plano como a criação
de um primeiro quadrado (o da sub-divisão de um canto, por exemplo)
e da translação de várias cópias deste quadrado de forma a obter um
quadrado maior constituído por menores quadrados.\newline
\break
\noindent
Esta transformação pode ser feita construíndo, no ínicio, uma linha
inferior de quadrados e multiplicando-o para cima, como pode ser visto
na Figura \ref{fig:plane}.

\begin{center}
    \includegraphics[width=0.6\textwidth]{imgs/plane.png}
    \captionof{figure}{Construção do plano}
    \label{fig:plane}
\end{center}

\subsubsection{Caixa}

Considerando o modelo da caixa, é rapidamente percétivel que esta pode ser
definida através de seis planos, aplicando-lhes rotações e translações.\newline
\break
\noindent
Esta definição torna-se, portanto, extremamente simples devido à já
existência de uma definição para plano, podendo ser definida pelas
transformações necessárias à criação da face superior e inferior e
posterior rotação de ambas para a criação das faces restantes.

\begin{center}
    \includegraphics[width=0.6\textwidth]{imgs/box.png}
    \captionof{figure}{Construção da caixa}
    \label{fig:box}
\end{center}

\subsubsection{Esfera}