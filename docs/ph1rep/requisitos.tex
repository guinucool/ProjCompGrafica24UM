\section{Requisitos}

Foram, portanto, especificados objetivos e funcionalidades pretendidas
para esta primeira fase, tanto para o motor gráfico como para o gerador
de primitivas.

\subsection{Gerador de Primitivas}

Numa fase inicial pretende-se que o gerador apenas seja capaz de gerar um
conjunto de quatro primitivas: \textbf{planos, caixas, esferas e cones}.
\newline
\break
\noindent
Estas primitivas deverão ser geradas dado, através de um comando,
um conjunto de propriedades que têm de respeitar e que as definem. Estas
propriedades podem ser o número de fatias, o número de pilhas, o tamanho
do lado, o raio, etc...\newline
\break
\noindent
Estas podem portanto ser obtidos através do seguintes comandos:

\begin{lstlisting}[style=BASH]
    $ generator plane length divisions path/to/file.3d
\end{lstlisting}

\begin{lstlisting}[style=BASH]
    $ generator box length divisions path/to/file.3d
\end{lstlisting}

\begin{lstlisting}[style=BASH]
    $ generator sphere radius slices stacks path/to/file.3d
\end{lstlisting}

\begin{lstlisting}[style=BASH]
    $ generator cone radius height slices stacks path/to/file.3d
\end{lstlisting}

\break
\noindent
Geradas as primitivas, as suas propriedades deverão ser armazenadas em
ficheiros \textit{3d}, dos quais deverá ser possível a reconstrução das
primitivas por parte do motor gráfico.

\subsection{Motor}

Já o motor, numa fase inicial, ficará encarregue, apenas, de ler e
interpretar uma configuração \textit{XML} que definirá um cenário
a ser representado.\newline
\break
\noindent
Esta configuração deverá ser do tipo:
\begin{tcolorbox}[
    colback=gray!10!white,
    colframe=black!50!black,
    after upper={\hfill\textbf{xml}}
]
\begin{verbatim}
<world>
    <window width="512" height="512" />
    <camera>
        <position x="3" y="2" z="1" />
        <lookAt x="0" y="0" z="0" />
        <up x="0" y="1" z="0" />
        <projection fov="60" near="1" far="1000" />
    </camera>
    <group>
        <models>
            <model file="plane.3d" />
            <model file="cone.3d" />
        </models>
    </group>
</world>
\end{verbatim}
\end{tcolorbox}

\break
\noindent
Em caso de parâmetros em falta que sejam substituíveis por um valor padrão,
essa deverá ser a decisão a tomar pelo interpretar \textit{XML} do motor
gráfico. O motor ainda deverá ser capaz de interpretar e armazenar ficheiros
\textit{3d}.
