\section{Requisitos}

Nesta secção serão especificadas as funcionalidades pretendidas  do \textbf{motor gráfico}
e do \textbf{gerador de primitivas}.

\subsection{Gerador}

O gerador de primitivas deve ser capaz de criar diversas \textbf{primitivas geométricas}, das quais
constam no enunciado \textit{esferas}, \textit{cubos}, \textit{cones} e \textit{planos}.
O conteúdo do output deste programa será um ficheiro com extensão \textit{.3d}
que contém os \textbf{pontos} estruturantes dos respetivos sólidos. De modo a que o motor possa
desenhar as figuras, a informação terá de ser organizada em \textbf{faces}, ou seja, triângulos.
Segue em exemplo um comando para gerar uma esfera de \textit{raio 1}, \textit{10 stacks, 10 fatias}
e guardar o resultado em \textit{sphere.3d}:
\begin{lstlisting}[style=BASH]
$ generator sphere 1 10 10 sphere.3d
\end{lstlisting}

\subsection{Motor}

O motor vai ler um \textbf{ficheiro de configuração \textit{.xml}}. Dessa configuração deverá
extrair as informações de \textbf{câmera}, \textbf{janela} e \textbf{modelos \textit{.3d}}.
Para este efeito é necessário implementar-se uma ferramenta de \textit{parsing xml}. Durante as
aulas teóricas desta UC foi-nos aconselhado o uso da biblioteca \textit{tinyxml2}.
Depois de processados os campos supracitados, o motor irá servir-se dos pontos presentes nos
\textbf{modelos} e das configurações para desenhar as figuras, através do \textit{OpenGL}.
