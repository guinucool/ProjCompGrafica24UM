Considerando os objetivos definidos, recolheu-se, como de costume,
um conjunto de requisitos que melhor descrevem os comportamentos
pretendidos para a solução final.

\subsection{Iluminação}

De forma a garantir uma boa implementação da componente
iluminatória, ambos os componentes de motor e gerador vão
precisar de ser atualizados.\\
\\
O gerador de primitivas deverá, portanto, ser capaz de gerar
normais para todos os pontos das figuras que constroi, armazenando-as
nos ficheiros \textit{.3d}. Estas
normais precisam de estar de acordo com o aspeto da figura de
forma a obter os melhores resultados, e é, por isso, importante
que seja o gerador a fazer essa atribuição.\\
\\
Já o motor gráfico deverá estar preparado para ler as normais
geradas e aplicá-las às respetivas figuras. Para além disso,
ainda deverá ser possível a criação de fontes de luz (de forma
a iluminar as figuras efetivamente).\\
\\
As fontes de luzes devem, ainda, ser semelhantes às figuras,
pertencendo a grupos e sub-grupos dos quais herdam transformações.
Porém, em constrate às figuras, uma fonte de luz apenas pode
iluminar os elementos do seu grupo e seus sub-grupos, pelo que
deverão ser configuradas com esta ideologia em mente.\\
\\
Isto permite aos utilizadores do programa a criação virtual
de mais luzes do que o \textbf{OpenGL} permite, ligando e
desligando luzes consoante os desenhos dos grupos que as pedem.
Ainda assim, não será possível ter mais do que oito luzes
ligadas em simultâneo.\\
\\
As fontes de luzes terão de ser definidas na configuração
\textbf{XML}, como, por exemplo:\\

\begin{tcolorbox}[
    colback=blue!10!white,
    colframe=black!50!black,
    after upper={\hfill\textbf{xml}}
]
\begin{verbatim}
<world> 
    <window width="512" height="512" />
    <camera>  
	    <position x="0" y="3" z="3" />
	    <lookAt x="0" y="0" z="0" />
	    <up x="0" y="1" z="0" />
        <projection fov="60" near="1" far="1000" /> 
    </camera>	
	<group>
        <lights> 
            <light
            type="spotlight" posX="0" posY="0" posZ="3"
			            dirX="0" dirY="0" dirZ="-1"
			            cutoff="5" />
        </lights>
		<group> 
			    <transform>
				        <rotate time="10" x="0" y="1" z="0" />
			    </transform>
			    <models> 
				        <model file="cone_1_2_4_3.3d" />
			    </models>
		</group>		
 	</group>
</world>
\end{verbatim}
\end{tcolorbox}

\subsection{Materiais e cores}
Tendo os objetos iluminados, será interessante ter, também,
cores diferentes para diferentes objetos.\\
\\
Para tal, pretende-se que seja possível definir as várias
componentes de cor de um modelo através da configuração
\textbf{XML} dada ao motor gráfico.\\
\\
Cada componente deverá ser definido através das suas intensidades
de \textbf{vermelho, verde e azul}, como pode ser observado em:\\

\begin{tcolorbox}[
    colback=blue!10!white,
    colframe=black!50!black,
    after upper={\hfill\textbf{xml}}
]
\begin{verbatim}
<models>
    <model file="sphere.3d" >
        <color>
            <diffuse R="200" G="200" B="200" />
            <ambient R="50" G="50" B="50" />
            <specular R="0" G="0" B="0" />
            <emissive R="0" G="0" B="0" />
            <shininess value="0" />
        </color>
    </model>
</models>
\end{verbatim}
\end{tcolorbox}

\subsection{Texturas}
Para além da coloração, pretende-se, ainda, que seja possível
aplicar texturas aos objetos.\\
\\
Esta tarefa estará, mais uma vez, dividida entre o gerador de
primitivas e o motor gráfico, ficando um deles encarregue de
gerar as coordenadas de textura para cada ponto de uma figura e, o 
outro, encarregue de carregar e aplicar texturas aos objetos
definidos anteriormente.\\
\\
As coordenadas de textura deverão, também, ser armazenadas
nos ficheiros \textit{.3d} e lidas e processadas pelo motor
gráfico.\\
\\
Uma textura deverá ser atribuida através do seguinte género
de configuração:\\

\begin{tcolorbox}[
    colback=blue!10!white,
    colframe=black!50!black,
    after upper={\hfill\textbf{xml}}
]
\begin{verbatim}
<models>
    <model file="sphere.3d" >
        <texture file="earth.jpg" />
    </model>
</models>
\end{verbatim}
\end{tcolorbox}