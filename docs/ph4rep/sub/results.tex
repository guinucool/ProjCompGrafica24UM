Implementada a solução final, foram testadas várias configurações
que permitissem demonstrar o poder do motor desenvolvido,
utilizando luzes, texturas, animações, etc...//
//
As cenas seguintes representam, então, as imagens criadas pelas
configurações dadas, que de certa forma refletem a capacidade
final do motor desenvolvido. Estas podem ser observadas em
\ref{fig:teapot}, \ref{fig:sisolar1} e \ref{fig:sisolar2}.

\subsection{Chaleira}

\begin{center}
    \includegraphics[width=0.8\textwidth]{imgs/teapot.png}
    \captionof{figure}{Chaleira construída com Bezier Patches}
    \label{fig:teapot}
\end{center}

\subsection{Sistema Solar num instante}

\begin{center}
    \includegraphics[width=0.8\textwidth]{imgs/sisolar1.png}
    \captionof{figure}{Sistema Solar animado num instante}
    \label{fig:sisolar1}
\end{center}

\subsection{Sistema Solar noutro instante}

\begin{center}
    \includegraphics[width=0.8\textwidth]{imgs/sisolar2.png}
    \captionof{figure}{Sistema Solar animado noutro instante}
    \label{fig:sisolar2}
\end{center}

