Embora já seja bastante percétivel através da arquitetura a
forma como as figuras, iluminação, texturas e cores vão ser
representadas, ainda é importante discutir a forma como as
normais e coordenadas de textura são calculadas.\\
\\
Cada figura tem as suas curvaturas e superfícies e, portanto,
é importante especificar como as texturas e normais se aplicam
a cada objeto usado.

\subsection{Normais}

\subsubsection{Plano e Cubo}

Sendo as figuras mais simples de calcular as normais de,
as normais destas figuras vão ser calculadas com a simples
perpendicularidade das faces destas figuras, já que as figuras
são quadradas.\\
\\
A normal de cada ponto será, portanto, simplesmente baseada
na normal do respetivo plano a que pertence.

\subsubsection{Esfera}

Sendo um pouco mais complicada, a esfera poderá ter as suas
normais calculadas através da simples extensão do raio da
esfera em si, uma vez que é o vetor mais prependicular que
existe em relação à curvatura da esfera naquele local.\\
\\
A coordenadas de cada normal serão, portanto, calculadas
através das coordenadas do vetor de distância (normalizadas)
entre a origem da esfera (que será a origem do
referencial) e o ponto da superfície.

\subsubsection{Cone}

Sendo a figura mais complicada, as normais do cone serão
calculadas através da perpendicularidade de um vetor com
a borda lateral do cone.\\
\\
Pretendendo se considerar o cone como uma figura circular e não
piramidal, esta aproximação pretende considerar a borda do cone
como se fosse uma "fatia" do mesmo.\\
\\
Para além disso, o cone precisará também de calcular as normais
para a sua base, que sendo uma superfície plana, serão semelhantes
às do plano e cubo.

\subsubsection{Patches de Bezier}

Por último, as normais de patches de bezier serão calculadas
através do cruzamento vetorial das suas derivadas, que, na sua
definição, já criam, desta forma, a respetiva normal do ponto.

\subsection{Texturas}

\subsubsection{Plano e Cubo}

Tendo o mapeamento basicamente idêntico ao das texturas,
as coordenadas de texturas destas figuras será uma simples
conversão das coordenadas para a escala de 0 a 1.

\subsubsection{Esfera}

Sendo a figura mais complicada para o mapeamento de texturas,
as coordenadas serão calculadas através do rebatimento das
suas coordenadas para um plano tangente.\\
\\
Este rebatimento permitirá uma melhor dispersão dos pontos nas
coordenadas de textura, tendo uma relação de distância e
dimensão mais precisa.

\subsubsection{Cone}

Tendo um mapeamento de texturas mais complicado, o cone será
mapeado através da sua altura e abertura do ângulo, para a sua
superfície lateral.\\
\\
As porpoções de ângulo serão, portanto, convertidas para a escala
horizontal e a altura para a escala vertical.\\
\\
A base será mapeada tal como o plano e o cubo.

\subsubsection{Patches de Bezier}

Por último, as coordenadas de textura das patches de bezier
serão calculadas através do uso dos parâmetros \textbf{u e v}
(mapeados para x e y) já usados para a criação das figuras,
que, apresentando já a escala certa, são perfeitos para o
propósito.