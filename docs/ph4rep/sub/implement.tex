Definida e estruturada a arquitetura da solução, foram, então,
especificadas soluções de implementação que tratassem as curvas
e os \textbf{VBO}s.

\subsection{Curvas e Superfícies}

Tendo em perpestiva a arquitetura e requisitos definidos,
assim como, a implementação das curvas, é possível perceber
que as ambas as curvas de \textbf{Catmull-Rom} e \textbf{Bezier}
podem ser implementadas em conjunto, sendo apenas necessário
alterar os valores da matriz de curva.
\newline
\break
\noindent
Assim sendo, foi feito uso do módulo de matriz, já definido na
primeira fase, que foi atualizado com novas operações relevantes
à operação com matrizes (transpostas, inversas, determinantes, etc...). 
\newline
\break
\noindent
Foram, então, definidas as matrizes de \textbf{Catmull-Rom} e \textbf{Bezier}
no módulo, que, em conjunto com as operações de curva e superfície, permitem
gerar posições e derivadas em curvas ou superfícies.
\newline
\break
\noindent
Estas operações são elementares de forma a que seja possível usar qualquer matriz de
curva para gerar posições e derivadas dado um instante da curva (ou dois, no caso da superfície)
e um conjunto de pontos de acordo.
\newline
\break
\noindent
Definidas as curvas e as suas operações, apenas é necessário, então, usar as posições
geradas por um intervalo de \textbf{Tesselation} para criar primitivas ou usar as
posições e derivadas geradas por um instante temporal para mover um objeto ao longo
de uma curva através de translações e rotações.

\subsection{Representação por VBO}

Considerando o modo de funcionamento dos \textbf{VBO} a solução escolhida para
a sua implementação permite o uso desta representação sem alterações de arquitetura.
\newline
\break
\noindent
Para tal, definiu-se que cada primitiva terá uma referência ao \textbf{buffer} responsável
por controlar a representação dos seus pontos.
\newline
\break
\noindent
Assim sendo, a representação de \textbf{VBO}s deverá ser definida por duas fases, uma para 
a alimentação das coordenadas dos pontos ao \textbf{buffer} e outra para a representação
destas coordenadas já definidas.
\newline
\break
\noindent
Cada ponto ficará, então, encarregue de alimentar as suas coordenadas ao \textbf{buffer}
da respetiva primitiva quando pedido para tal.
\newline
\break
\noindent
Alimentados todos os \textbf{buffers} de todas as primitivas, apenas será necessária a
sua posterior representação através dos \textbf{buffers} definidos.