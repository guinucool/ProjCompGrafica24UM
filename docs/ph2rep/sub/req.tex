Definidos os objetivos para esta fase da implementação,
especificaram-se requisitos, garantias e funcionalidades
que a arquitetura deverá respeitar.
\newline
\break
\noindent
Em constrate com a fase inicial desta solução, para
garantir a implementação pretendida apenas será
necessário atualizar o motor gráfico, que agora deverá
conseguir identificar transformações e sub-grupos.

\subsection{Transformações}

O motor gŕafico terá, portanto, a necessidade de
conseguir identificar e armazenar todos os tipos
de transformações existentes.
\newline
\break
\noindent
Existem três tipos de transformações:
\textbf{translações, rotações e escalas} e cada uma delas
apresenta um conjunto de propriedades que permitem
definir como a primitiva será afeta por elas.
\newline

\begin{tcolorbox}[
    colback=blue!10!white,
    colframe=black!50!black,
]
\begin{verbatim}
Translação:
    x, y, z -> translação sobre os eixos x, y, z

Rotação:
    angle -> ângulo de rotação
    x, y, z -> eixo de rotação definido

Scale:
    x, y, z -> escala sobre os eixos x, y, z
\end{verbatim}
\end{tcolorbox}

\vspace{12pt}
\noindent
Estas transformações devem, então, ser definidas numa
configuração \textbf{XML} envolvidas sob uma etiqueta
\textbf{Transform}, que inicia um grupo de
transformações.
\newline
\break
\noindent
Dentro de um grupo não é permitido, também,
existir repetições de um mesmo tipo de transformação,
ou seja, se um grupo já possuí uma translação, uma outra
translação apenas pode ser definida noutro grupo ou
sub-grupo.
\newline
\break
\noindent
A configuração assume, portanto, o seguinte
formato exemplar:
\newline

\begin{tcolorbox}[
    colback=blue!10!white,
    colframe=black!50!black,
    after upper={\hfill\textbf{xml}}
]
\begin{verbatim}
<world>
    <window width="512" height="512" />
    <camera>
        <position x="10" y="10" z="10" />
        <lookAt x="0" y="0" z="0" />
        <up x="0" y="1" z="0" />
        <projection fov="60" near="1" far="1000" />
    </camera>
    <group>
        <transform>
            <translate x="4" y="0" z="0" />
            <rotate angle="30" x="0" y="1" z="0" />
            <scale x="2" y="0.3" z="1" />
        </transform>
        <models>
            <model file="cone.3d" />
            <model file="plane.3d" />
        </models>
    </group>
</world>
\end{verbatim}
\end{tcolorbox}